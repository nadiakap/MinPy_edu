\documentclass[12pt]{article}
\usepackage[a4paper,margin=1in]{geometry}
\usepackage{amsmath,amssymb,amsfonts}
\usepackage{bm}
\usepackage{physics}
\usepackage{algorithm}
\usepackage{algorithmic}
\usepackage{graphicx}
\usepackage{hyperref}

\title{\textbf{Documentation and Mathematical Foundations of the \texttt{minpy} Optimization Framework}}
\author{Nadia Udler}
\date{\today}

\begin{document}

\maketitle

\section{Overview}

The \texttt{minpy} framework implements a family of nonlocal optimization algorithms based on \emph{potential theory}, following the ideas in Kaplinskii \& Propoi (``Nonlocal Optimization Methods Based on Potential Theory'').  It includes:

\begin{itemize}
    \item A \textbf{potential-based first-order method}, which constructs a potential field from sampled points and moves particles along the gradient of this potential.
    \item A \textbf{second-order potential method}, which uses local curvature (Hessian) information to accelerate convergence.
    \item Classical methods (particle swarm, Nelder–Mead, etc.) available in the same interface for comparison.
\end{itemize}

Each method can handle multimodal and nonconvex objectives by combining global exploration and local refinement.

\section{Mathematical Foundation}

\subsection{Potential Function Formulation}

Given an objective function $f: \mathbb{R}^d \to \mathbb{R}$, we maintain a set of sample points (``particles'')
\[
U = \{ \mathbf{u}_1, \mathbf{u}_2, \ldots, \mathbf{u}_K \}, \qquad f_i = f(\mathbf{u}_i).
\]
From these samples, we construct a potential function
\[
\Phi(\mathbf{x}) = \sum_{j=1}^K w_j \, G\!\left(\|\mathbf{x} - \mathbf{u}_j\|; \varepsilon\right),
\]
where:
\begin{itemize}
    \item $G(r; \varepsilon)$ is a \emph{radial kernel} that smooths the field,
    \item $w_j$ are weights determined by the quality of the sample,
    \item $\varepsilon > 0$ controls the range of influence (annealed over iterations).
\end{itemize}

Typical choices for $G$ include:
\begin{align}
    G_\text{inv-power}(r) &= \frac{1}{(r^p + \varepsilon)}, \\
    G_\text{inv-mq}(r) &= \frac{1}{\sqrt{r^2 + \varepsilon}}, \\
    G_\text{gauss}(r) &= \exp\!\left(-\frac{r^2}{2\sigma^2}\right).
\end{align}

\subsection{Potential Gradient and Particle Movement}

The ``force'' acting on a particle at $\mathbf{u}_i$ is given by the negative gradient of the potential:
\[
\mathbf{F}_i = -\nabla_{\mathbf{x}} \Phi(\mathbf{x}) \big|_{\mathbf{x} = \mathbf{u}_i}
= - \sum_{j=1}^K w_j \, G'(r_{ij}) \, \frac{\mathbf{u}_i - \mathbf{u}_j}{r_{ij}},
\]
where $r_{ij} = \|\mathbf{u}_i - \mathbf{u}_j\|$ and $G'(r)$ is the radial derivative of the kernel.

The particles move along this field:
\[
\mathbf{u}_i^{(t+1)} = \mathbf{u}_i^{(t)} + \eta_t \, \mathbf{F}_i^{(t)},
\]
where $\eta_t$ is a step size controlling the motion amplitude.

\subsection{Weight Adaptation}

Weights $w_j$ emphasize points with lower function values:
\[
w_j = \frac{\exp(-\beta_t (f_j - f_{\min}))}{\sum_{k=1}^K \exp(-\beta_t (f_k - f_{\min}))}.
\]
Here $\beta_t$ acts as an inverse temperature parameter that increases over iterations (annealing), focusing attraction toward promising regions.

\subsection{Nonlocal--Local Transition}

The potential parameter $\varepsilon_t$ is decreased according to an annealing schedule:
\[
\varepsilon_{t+1} = \rho_\varepsilon \, \varepsilon_t, \qquad 0 < \rho_\varepsilon < 1,
\]
which gradually sharpens the potential field from a smooth global surface to a fine local structure.

\section{Algorithm 1: Potential-Based Nonlocal Minimization}

\begin{algorithm}[h!]
\caption{Nonlocal Potential Optimization (First-Order)}
\begin{algorithmic}[1]
\STATE Initialize $K$ particles $\mathbf{u}_i$ within bounds $[\mathbf{l}, \mathbf{u}]$.
\STATE Evaluate $f_i = f(\mathbf{u}_i)$.
\STATE Set $\varepsilon \leftarrow \varepsilon_0$, $\beta \leftarrow \beta_0$.
\FOR{epoch $t = 1, \ldots, T$}
    \STATE Compute weights $w_j$ from $f_j$.
    \STATE For each particle $i$, compute force
    \[
        \mathbf{F}_i = -\sum_j w_j G'(r_{ij}) \frac{\mathbf{u}_i - \mathbf{u}_j}{r_{ij}}.
    \]
    \STATE Update positions $\mathbf{u}_i \leftarrow \mathbf{u}_i + \eta \, \mathbf{F}_i$.
    \STATE Apply bounds and recompute $f_i$.
    \STATE Anneal parameters:
    $\varepsilon \leftarrow \rho_\varepsilon \varepsilon$, \quad $\beta \leftarrow \rho_\beta \beta$.
\ENDFOR
\STATE Return best particle $\mathbf{u}_\ast = \arg\min_i f_i$.
\end{algorithmic}
\end{algorithm}

This algorithm is implemented in \texttt{minimize\_potential\_nonlocal()}.

\section{Algorithm 2: Second-Order Potential Optimization}

To accelerate convergence near minima, we augment the previous scheme with local second-order updates (Newton or Levenberg–Marquardt type).

\subsection{Gradient and Hessian Approximation}

For a point $\mathbf{x}$, we approximate:
\[
\nabla f(\mathbf{x})_i \approx \frac{f(\mathbf{x}+h\mathbf{e}_i) - f(\mathbf{x}-h\mathbf{e}_i)}{2h},
\]
and
\[
H_{ij}(\mathbf{x}) \approx \frac{f(\mathbf{x}+h(\mathbf{e}_i+\mathbf{e}_j)) - f(\mathbf{x}+h(\mathbf{e}_i-\mathbf{e}_j)) - f(\mathbf{x}-h(\mathbf{e}_i-\mathbf{e}_j)) + f(\mathbf{x}-h(\mathbf{e}_i+\mathbf{e}_j))}{4h^2}.
\]

\subsection{Damped Newton Step}

The Newton update is computed as:
\[
\Delta \mathbf{x} = - (H + \lambda I)^{-1} \nabla f,
\]
where $\lambda > 0$ is a damping factor.  
If the step fails to reduce $f$, $\lambda$ is increased; otherwise it is decreased.

\begin{algorithm}[h!]
\caption{Potential + Second-Order Refinement}
\begin{algorithmic}[1]
\STATE Run potential-based updates as in Algorithm 1.
\STATE Select $N_\text{top}$ best particles for local refinement.
\FOR{each selected particle $\mathbf{u}_i$}
    \FOR{Newton iteration $k=1,\ldots,N_\text{newton}$}
        \STATE Compute gradient $\mathbf{g}$ and Hessian $H$.
        \STATE Solve $(H+\lambda I)\Delta \mathbf{x} = -\mathbf{g}$.
        \STATE Line-search along $\Delta \mathbf{x}$ to ensure decrease in $f$.
        \STATE If improvement $< \tau$, increase $\lambda$.
    \ENDFOR
\ENDFOR
\STATE Return updated best particle.
\end{algorithmic}
\end{algorithm}

This corresponds to \texttt{minimize\_potential\_second\_order()} in the Python code.

\section{Connection to Classical Optimization Methods}

\subsection{Relation to Momentum Methods}

In first-order gradient methods with momentum:
\[
\mathbf{v}_{t+1} = \mu \mathbf{v}_t - \eta \nabla f(\mathbf{x}_t), \qquad
\mathbf{x}_{t+1} = \mathbf{x}_t + \mathbf{v}_{t+1}.
\]
Kaplinskii’s potential methods can be viewed as a \emph{discretized, collective} momentum system, where multiple trajectories share information through a smoothed potential rather than direct gradient terms.

\subsection{Relation to Particle Swarm Optimization}

The potential force term $\mathbf{F}_i$ plays the same role as the attraction toward global and local best in PSO, but arises from a physically motivated potential rather than heuristic coefficients.

\subsection{Relation to Newton and Quasi-Newton Methods}

The second-order method approximates the local curvature of the objective directly, yielding steps similar to Newton or Levenberg–Marquardt updates:
\[
\mathbf{x}_{t+1} = \mathbf{x}_t - (H_t + \lambda I)^{-1} \nabla f_t.
\]
In practice, this allows quadratic convergence once the algorithm enters a local basin of attraction.

\section{Implementation Structure in Python}

\subsection{Class \texttt{Minimization}}
\begin{itemize}
    \item \textbf{Attributes:}
        \begin{itemize}
            \item \texttt{u} : array of shape $(K, d)$, particle coordinates.
            \item \texttt{fu} : array of objective values.
            \item \texttt{lb, ub} : lower and upper bounds.
            \item \texttt{dim} : problem dimension.
        \end{itemize}
    \item \textbf{Methods:}
        \begin{itemize}
            \item \texttt{get\_f(x)}: evaluates the objective.
            \item \texttt{sift()}: resamples poor particles.
            \item \texttt{minimize\_potential\_nonlocal()}: first-order method.
            \item \texttt{minimize\_potential\_second\_order()}: second-order refinement.
        \end{itemize}
\end{itemize}

\subsection{Usage Example}

\begin{verbatim}
def ackley(x):
    x = np.asarray(x)
    d = len(x)
    term1 = -20 * np.exp(-0.2 * np.sqrt(np.sum(x**2)/d))
    term2 = -np.exp(np.sum(np.cos(2*np.pi*x))/d)
    return term1 + term2 + 20 + np.e

m = Minimization(ackley, X0=np.zeros(2), K=60, lb=[-5,-5], ub=[5,5])
fbest, xbest = m.minimize_potential_second_order(verbose=True)
\end{verbatim}

\section{Parameter Summary}

\begin{center}
\begin{tabular}{l l l}
\hline
Parameter & Description & Typical Range \\
\hline
$\varepsilon_0$ & Initial potential smoothing & $1 \text{--} 5$ \\
$\varepsilon_\text{final}$ & Final smoothing scale & $10^{-4}$ \\
$\beta_0$ & Initial inverse temperature & $0.5$ \\
$\beta_\text{final}$ & Final inverse temperature & $20\text{--}50$ \\
$\eta$ & Step size & $0.1\text{--}0.5$ \\
$K$ & Number of particles & $30\text{--}200$ \\
$\lambda$ & Damping for Newton step & $10^{-4}\text{--}10^{-2}$ \\
\hline
\end{tabular}
\end{center}

\section{Convergence Considerations}

Under mild smoothness conditions, the potential-based method converges toward regions where $\nabla f \approx 0$, and the second-order refinement ensures rapid local convergence near stationary points.

Potential-based algorithms exhibit good robustness in multimodal landscapes due to the nonlocal interactions among particles, which prevent premature convergence to poor minima.

\section{References}

\begin{enumerate}
    \item Kaplinskii, I., \& Propoi, A. (1994). \emph{Nonlocal Optimization Methods Based on Potential Theory.}
    \item Vidyasagar, M. (2005). \emph{Theory of Learning and Generalization: With Applications to Neural Networks and Control Systems.} Springer.
    \item Nesterov, Y. (2004). \emph{Introductory Lectures on Convex Optimization: A Basic Course.} Kluwer.
\end{enumerate}

\end{document}
